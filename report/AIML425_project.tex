%  Template for ICASSP-2021 paper; to be used with:
%          spconf.sty  - ICASSP/ICIP LaTeX style file, and
%          IEEEbib.bst - IEEE bibliography style file.
% --------------------------------------------------------------------------
\documentclass{article}
\usepackage{spconf,amsmath,graphicx,hyperref}

% Example definitions.
% --------------------
\def\x{{\mathbf x}}
\def\L{{\cal L}}

% Title.
% ------
\title{AIML 425 Project}
%
% Single address.
% ---------------
\name{Quan Zhao (Student ID: 300471028)}
%\name{Author(s) Name(s)\thanks{Thanks to XYZ agency for funding.}}
\address{Victoria University of Wellington}


\begin{document}
%\ninept
%
\maketitle
%
\section{Introduction}
\label{sec:intro}

Face Detection is a very popular research topic and have various applications in industry.
Nerual Network plays key role in this work, lots of Face Detection Nerual Networks has been introduced.
RetinaFace is one of most populars implementation, which is been introduced in 2019, 
and be accepted by "Conference on Computer Vision and Pattern Recognition (CVPR)" in 2020.
In this study, I will show my understanding of this work based on this paper \cite{deng2020retinaface}
In real world, lots of edge case can not be covered by this pretrained model.
Retrain whole model cost too much.
Fine tuning model approach will be introduced in this study.


\section{THEORY}
\label{sec:theory}

\subsection{RetinaNet}

\subsection{feature pyramid network (FPN)}

\subsection{Context Modelling}

\subsection{Multi-task Loss}

\subsection{Anchor}

\subsection{mAP}


\section{RESULTS}
\label{sec:results}

\subsection{Data Preparation}
\label{ssec:data}


  \subsection{Model Fine Tuning and evaluation}
  \label{ssec:model}

\section{Impact}
  
\section{CONCLUSION}
\label{sec:conclusion}


\section{STATEMENT OF ALL TOOLS USED}
\label{sec:statementofalltoolsused}

% In this work, we used Pytorch geometric package to generate data, create, train models. 
% The Plotly package helped us visualize in 3D. 

Source codes are published in github: 
% $\href{}{URL}$




% To start a new column (but not a new page) and help balance the last-page
% column length use \vfill\pagebreak.
% -------------------------------------------------------------------------
%\vfill
%\pagebreak

\vfill\pagebreak

% References should be produced using the bibtex program from suitable
% BiBTeX files (here: strings, refs, manuals). The IEEEbib.bst bibliography
% style file from IEEE produces unsorted bibliography list.
% -------------------------------------------------------------------------
\bibliographystyle{IEEEbib}
\bibliography{strings,refs}

\end{document}